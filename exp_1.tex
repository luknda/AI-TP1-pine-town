% Options for packages loaded elsewhere
\PassOptionsToPackage{unicode}{hyperref}
\PassOptionsToPackage{hyphens}{url}
%
\documentclass[
]{article}
\usepackage{amsmath,amssymb}
\usepackage{iftex}
\ifPDFTeX
  \usepackage[T1]{fontenc}
  \usepackage[utf8]{inputenc}
  \usepackage{textcomp} % provide euro and other symbols
\else % if luatex or xetex
  \usepackage{unicode-math} % this also loads fontspec
  \defaultfontfeatures{Scale=MatchLowercase}
  \defaultfontfeatures[\rmfamily]{Ligatures=TeX,Scale=1}
\fi
\usepackage{lmodern}
\ifPDFTeX\else
  % xetex/luatex font selection
\fi
% Use upquote if available, for straight quotes in verbatim environments
\IfFileExists{upquote.sty}{\usepackage{upquote}}{}
\IfFileExists{microtype.sty}{% use microtype if available
  \usepackage[]{microtype}
  \UseMicrotypeSet[protrusion]{basicmath} % disable protrusion for tt fonts
}{}
\makeatletter
\@ifundefined{KOMAClassName}{% if non-KOMA class
  \IfFileExists{parskip.sty}{%
    \usepackage{parskip}
  }{% else
    \setlength{\parindent}{0pt}
    \setlength{\parskip}{6pt plus 2pt minus 1pt}}
}{% if KOMA class
  \KOMAoptions{parskip=half}}
\makeatother
\usepackage{xcolor}
\usepackage[margin=1in]{geometry}
\usepackage{color}
\usepackage{fancyvrb}
\newcommand{\VerbBar}{|}
\newcommand{\VERB}{\Verb[commandchars=\\\{\}]}
\DefineVerbatimEnvironment{Highlighting}{Verbatim}{commandchars=\\\{\}}
% Add ',fontsize=\small' for more characters per line
\usepackage{framed}
\definecolor{shadecolor}{RGB}{248,248,248}
\newenvironment{Shaded}{\begin{snugshade}}{\end{snugshade}}
\newcommand{\AlertTok}[1]{\textcolor[rgb]{0.94,0.16,0.16}{#1}}
\newcommand{\AnnotationTok}[1]{\textcolor[rgb]{0.56,0.35,0.01}{\textbf{\textit{#1}}}}
\newcommand{\AttributeTok}[1]{\textcolor[rgb]{0.13,0.29,0.53}{#1}}
\newcommand{\BaseNTok}[1]{\textcolor[rgb]{0.00,0.00,0.81}{#1}}
\newcommand{\BuiltInTok}[1]{#1}
\newcommand{\CharTok}[1]{\textcolor[rgb]{0.31,0.60,0.02}{#1}}
\newcommand{\CommentTok}[1]{\textcolor[rgb]{0.56,0.35,0.01}{\textit{#1}}}
\newcommand{\CommentVarTok}[1]{\textcolor[rgb]{0.56,0.35,0.01}{\textbf{\textit{#1}}}}
\newcommand{\ConstantTok}[1]{\textcolor[rgb]{0.56,0.35,0.01}{#1}}
\newcommand{\ControlFlowTok}[1]{\textcolor[rgb]{0.13,0.29,0.53}{\textbf{#1}}}
\newcommand{\DataTypeTok}[1]{\textcolor[rgb]{0.13,0.29,0.53}{#1}}
\newcommand{\DecValTok}[1]{\textcolor[rgb]{0.00,0.00,0.81}{#1}}
\newcommand{\DocumentationTok}[1]{\textcolor[rgb]{0.56,0.35,0.01}{\textbf{\textit{#1}}}}
\newcommand{\ErrorTok}[1]{\textcolor[rgb]{0.64,0.00,0.00}{\textbf{#1}}}
\newcommand{\ExtensionTok}[1]{#1}
\newcommand{\FloatTok}[1]{\textcolor[rgb]{0.00,0.00,0.81}{#1}}
\newcommand{\FunctionTok}[1]{\textcolor[rgb]{0.13,0.29,0.53}{\textbf{#1}}}
\newcommand{\ImportTok}[1]{#1}
\newcommand{\InformationTok}[1]{\textcolor[rgb]{0.56,0.35,0.01}{\textbf{\textit{#1}}}}
\newcommand{\KeywordTok}[1]{\textcolor[rgb]{0.13,0.29,0.53}{\textbf{#1}}}
\newcommand{\NormalTok}[1]{#1}
\newcommand{\OperatorTok}[1]{\textcolor[rgb]{0.81,0.36,0.00}{\textbf{#1}}}
\newcommand{\OtherTok}[1]{\textcolor[rgb]{0.56,0.35,0.01}{#1}}
\newcommand{\PreprocessorTok}[1]{\textcolor[rgb]{0.56,0.35,0.01}{\textit{#1}}}
\newcommand{\RegionMarkerTok}[1]{#1}
\newcommand{\SpecialCharTok}[1]{\textcolor[rgb]{0.81,0.36,0.00}{\textbf{#1}}}
\newcommand{\SpecialStringTok}[1]{\textcolor[rgb]{0.31,0.60,0.02}{#1}}
\newcommand{\StringTok}[1]{\textcolor[rgb]{0.31,0.60,0.02}{#1}}
\newcommand{\VariableTok}[1]{\textcolor[rgb]{0.00,0.00,0.00}{#1}}
\newcommand{\VerbatimStringTok}[1]{\textcolor[rgb]{0.31,0.60,0.02}{#1}}
\newcommand{\WarningTok}[1]{\textcolor[rgb]{0.56,0.35,0.01}{\textbf{\textit{#1}}}}
\usepackage{graphicx}
\makeatletter
\def\maxwidth{\ifdim\Gin@nat@width>\linewidth\linewidth\else\Gin@nat@width\fi}
\def\maxheight{\ifdim\Gin@nat@height>\textheight\textheight\else\Gin@nat@height\fi}
\makeatother
% Scale images if necessary, so that they will not overflow the page
% margins by default, and it is still possible to overwrite the defaults
% using explicit options in \includegraphics[width, height, ...]{}
\setkeys{Gin}{width=\maxwidth,height=\maxheight,keepaspectratio}
% Set default figure placement to htbp
\makeatletter
\def\fps@figure{htbp}
\makeatother
\setlength{\emergencystretch}{3em} % prevent overfull lines
\providecommand{\tightlist}{%
  \setlength{\itemsep}{0pt}\setlength{\parskip}{0pt}}
\setcounter{secnumdepth}{-\maxdimen} % remove section numbering
\ifLuaTeX
  \usepackage{selnolig}  % disable illegal ligatures
\fi
\IfFileExists{bookmark.sty}{\usepackage{bookmark}}{\usepackage{hyperref}}
\IfFileExists{xurl.sty}{\usepackage{xurl}}{} % add URL line breaks if available
\urlstyle{same}
\hypersetup{
  pdftitle={Untitled},
  hidelinks,
  pdfcreator={LaTeX via pandoc}}

\title{Untitled}
\author{}
\date{\vspace{-2.5em}2023-08-21}

\begin{document}
\maketitle

\begin{Shaded}
\begin{Highlighting}[]
\FunctionTok{source}\NormalTok{(}\StringTok{"C:}\SpecialCharTok{\textbackslash{}\textbackslash{}}\StringTok{Users}\SpecialCharTok{\textbackslash{}\textbackslash{}}\StringTok{crash}\SpecialCharTok{\textbackslash{}\textbackslash{}}\StringTok{OneDrive}\SpecialCharTok{\textbackslash{}\textbackslash{}}\StringTok{Documents}\SpecialCharTok{\textbackslash{}\textbackslash{}}\StringTok{AI{-}TP1{-}pine{-}town}\SpecialCharTok{\textbackslash{}\textbackslash{}}\StringTok{exp\_1.R"}\NormalTok{)}
\end{Highlighting}
\end{Shaded}

\begin{verbatim}
## 
## Attaching package: 'dplyr'
\end{verbatim}

\begin{verbatim}
## The following objects are masked from 'package:stats':
## 
##     filter, lag
\end{verbatim}

\begin{verbatim}
## The following objects are masked from 'package:base':
## 
##     intersect, setdiff, setequal, union
\end{verbatim}

\begin{verbatim}
## Loading required package: lattice
\end{verbatim}

\begin{verbatim}
## 
## Attaching package: 'caret'
\end{verbatim}

\begin{verbatim}
## The following objects are masked from 'package:Metrics':
## 
##     precision, recall
\end{verbatim}

\begin{verbatim}
## Loading required package: iterators
\end{verbatim}

\begin{verbatim}
## Loading required package: parallel
\end{verbatim}

\begin{verbatim}
## [1] "Churn" "Yes"   "0.1"  
## [1] 1
## [1] 2
## [1] 3
## [1] 4
## [1] 5
## [1] 6
## [1] 7
## [1] 8
## [1] 9
## [1] 10
## [1] 11
## [1] 12
## [1] 13
## [1] 14
## [1] 15
## [1] 16
## [1] 17
## [1] 18
## [1] 19
## [1] 20
## [1] 21
## [1] 22
## [1] 23
## [1] 24
## [1] 25
## [1] 26
## [1] 27
## [1] 28
## [1] 29
## [1] 30
## [1] "Churn" "Yes"   "0.3"  
## [1] 1
## [1] 2
## [1] 3
## [1] 4
## [1] 5
## [1] 6
## [1] 7
## [1] 8
## [1] 9
## [1] 10
## [1] 11
## [1] 12
## [1] 13
## [1] 14
## [1] 15
## [1] 16
## [1] 17
## [1] 18
## [1] 19
## [1] 20
## [1] 21
## [1] 22
## [1] 23
## [1] 24
## [1] 25
## [1] 26
## [1] 27
## [1] 28
## [1] 29
## [1] 30
## [1] "Churn" "Yes"   "0.5"  
## [1] 1
## [1] 2
## [1] 3
## [1] 4
## [1] 5
## [1] 6
## [1] 7
## [1] 8
## [1] 9
## [1] 10
## [1] 11
## [1] 12
## [1] 13
## [1] 14
## [1] 15
## [1] 16
## [1] 17
## [1] 18
## [1] 19
## [1] 20
## [1] 21
## [1] 22
## [1] 23
## [1] 24
## [1] 25
## [1] 26
## [1] 27
## [1] 28
## [1] 29
## [1] 30
## [1] "Churn" "Yes"   "0.7"  
## [1] 1
## [1] 2
## [1] 3
## [1] 4
## [1] 5
## [1] 6
## [1] 7
## [1] 8
## [1] 9
## [1] 10
## [1] 11
## [1] 12
## [1] 13
## [1] 14
## [1] 15
## [1] 16
## [1] 17
## [1] 18
## [1] 19
## [1] 20
## [1] 21
## [1] 22
## [1] 23
## [1] 24
## [1] 25
## [1] 26
## [1] 27
## [1] 28
## [1] 29
## [1] 30
## [1] "Churn" "Yes"   "0.9"  
## [1] 1
## [1] 2
## [1] 3
## [1] 4
## [1] 5
## [1] 6
## [1] 7
## [1] 8
## [1] 9
## [1] 10
## [1] 11
## [1] 12
## [1] 13
## [1] 14
## [1] 15
## [1] 16
## [1] 17
## [1] 18
## [1] 19
## [1] 20
## [1] 21
## [1] 22
## [1] 23
## [1] 24
## [1] 25
## [1] 26
## [1] 27
## [1] 28
## [1] 29
## [1] 30
## [1] "Churn" "No"    "0.1"  
## [1] 1
## [1] 2
## [1] 3
## [1] 4
## [1] 5
## [1] 6
## [1] 7
## [1] 8
## [1] 9
## [1] 10
## [1] 11
## [1] 12
## [1] 13
## [1] 14
## [1] 15
## [1] 16
## [1] 17
## [1] 18
## [1] 19
## [1] 20
## [1] 21
## [1] 22
## [1] 23
## [1] 24
## [1] 25
## [1] 26
## [1] 27
## [1] 28
## [1] 29
## [1] 30
## [1] "Churn" "No"    "0.3"  
## [1] 1
## [1] 2
## [1] 3
## [1] 4
## [1] 5
## [1] 6
## [1] 7
## [1] 8
## [1] 9
## [1] 10
## [1] 11
## [1] 12
## [1] 13
## [1] 14
## [1] 15
## [1] 16
## [1] 17
## [1] 18
## [1] 19
## [1] 20
## [1] 21
## [1] 22
## [1] 23
## [1] 24
## [1] 25
## [1] 26
## [1] 27
## [1] 28
## [1] 29
## [1] 30
## [1] "Churn" "No"    "0.5"  
## [1] 1
## [1] 2
## [1] 3
## [1] 4
## [1] 5
## [1] 6
## [1] 7
## [1] 8
## [1] 9
## [1] 10
## [1] 11
## [1] 12
## [1] 13
## [1] 14
## [1] 15
## [1] 16
## [1] 17
## [1] 18
## [1] 19
## [1] 20
## [1] 21
## [1] 22
## [1] 23
## [1] 24
## [1] 25
## [1] 26
## [1] 27
## [1] 28
## [1] 29
## [1] 30
## [1] "Churn" "No"    "0.7"  
## [1] 1
## [1] 2
## [1] 3
## [1] 4
## [1] 5
## [1] 6
## [1] 7
## [1] 8
## [1] 9
## [1] 10
## [1] 11
## [1] 12
## [1] 13
## [1] 14
## [1] 15
## [1] 16
## [1] 17
## [1] 18
## [1] 19
## [1] 20
## [1] 21
## [1] 22
## [1] 23
## [1] 24
## [1] 25
## [1] 26
## [1] 27
## [1] 28
## [1] 29
## [1] 30
## [1] "Churn" "No"    "0.9"  
## [1] 1
## [1] 2
## [1] 3
## [1] 4
## [1] 5
## [1] 6
## [1] 7
## [1] 8
## [1] 9
## [1] 10
## [1] 11
## [1] 12
## [1] 13
## [1] 14
## [1] 15
## [1] 16
## [1] 17
## [1] 18
## [1] 19
## [1] 20
## [1] 21
## [1] 22
## [1] 23
## [1] 24
## [1] 25
## [1] 26
## [1] 27
## [1] 28
## [1] 29
## [1] 30
## [1] "Heart" "Yes"   "0.1"  
## [1] 1
## [1] 2
## [1] 3
## [1] 4
## [1] 5
## [1] 6
## [1] 7
## [1] 8
## [1] 9
## [1] 10
## [1] 11
## [1] 12
## [1] 13
## [1] 14
## [1] 15
## [1] 16
## [1] 17
## [1] 18
## [1] 19
## [1] 20
## [1] 21
## [1] 22
## [1] 23
## [1] 24
## [1] 25
## [1] 26
## [1] 27
## [1] 28
## [1] 29
## [1] 30
## [1] "Heart" "Yes"   "0.3"  
## [1] 1
## [1] 2
## [1] 3
## [1] 4
## [1] 5
## [1] 6
## [1] 7
## [1] 8
## [1] 9
## [1] 10
## [1] 11
## [1] 12
## [1] 13
## [1] 14
## [1] 15
## [1] 16
## [1] 17
## [1] 18
## [1] 19
## [1] 20
## [1] 21
## [1] 22
## [1] 23
## [1] 24
## [1] 25
## [1] 26
## [1] 27
## [1] 28
## [1] 29
## [1] 30
## [1] "Heart" "Yes"   "0.5"  
## [1] 1
## [1] 2
## [1] 3
## [1] 4
## [1] 5
## [1] 6
## [1] 7
## [1] 8
## [1] 9
## [1] 10
## [1] 11
## [1] 12
## [1] 13
## [1] 14
## [1] 15
## [1] 16
## [1] 17
## [1] 18
## [1] 19
## [1] 20
## [1] 21
## [1] 22
## [1] 23
## [1] 24
## [1] 25
## [1] 26
## [1] 27
## [1] 28
## [1] 29
## [1] 30
## [1] "Heart" "Yes"   "0.7"  
## [1] 1
## [1] 2
## [1] 3
## [1] 4
## [1] 5
## [1] 6
## [1] 7
## [1] 8
## [1] 9
## [1] 10
## [1] 11
## [1] 12
## [1] 13
## [1] 14
## [1] 15
## [1] 16
## [1] 17
## [1] 18
## [1] 19
## [1] 20
## [1] 21
## [1] 22
## [1] 23
## [1] 24
## [1] 25
## [1] 26
## [1] 27
## [1] 28
## [1] 29
## [1] 30
## [1] "Heart" "Yes"   "0.9"  
## [1] 1
## [1] 2
## [1] 3
## [1] 4
## [1] 5
## [1] 6
## [1] 7
## [1] 8
## [1] 9
## [1] 10
## [1] 11
## [1] 12
## [1] 13
## [1] 14
## [1] 15
## [1] 16
## [1] 17
## [1] 18
## [1] 19
## [1] 20
## [1] 21
## [1] 22
## [1] 23
## [1] 24
## [1] 25
## [1] 26
## [1] 27
## [1] 28
## [1] 29
## [1] 30
## [1] "Heart" "No"    "0.1"  
## [1] 1
## [1] 2
## [1] 3
## [1] 4
## [1] 5
## [1] 6
## [1] 7
## [1] 8
## [1] 9
## [1] 10
## [1] 11
## [1] 12
## [1] 13
## [1] 14
## [1] 15
## [1] 16
## [1] 17
## [1] 18
## [1] 19
## [1] 20
## [1] 21
## [1] 22
## [1] 23
## [1] 24
## [1] 25
## [1] 26
## [1] 27
## [1] 28
## [1] 29
## [1] 30
## [1] "Heart" "No"    "0.3"  
## [1] 1
## [1] 2
## [1] 3
## [1] 4
## [1] 5
## [1] 6
## [1] 7
## [1] 8
## [1] 9
## [1] 10
## [1] 11
## [1] 12
## [1] 13
## [1] 14
## [1] 15
## [1] 16
## [1] 17
## [1] 18
## [1] 19
## [1] 20
## [1] 21
## [1] 22
## [1] 23
## [1] 24
## [1] 25
## [1] 26
## [1] 27
## [1] 28
## [1] 29
## [1] 30
## [1] "Heart" "No"    "0.5"  
## [1] 1
## [1] 2
## [1] 3
## [1] 4
## [1] 5
## [1] 6
## [1] 7
## [1] 8
## [1] 9
## [1] 10
## [1] 11
## [1] 12
## [1] 13
## [1] 14
## [1] 15
## [1] 16
## [1] 17
## [1] 18
## [1] 19
## [1] 20
## [1] 21
## [1] 22
## [1] 23
## [1] 24
## [1] 25
## [1] 26
## [1] 27
## [1] 28
## [1] 29
## [1] 30
## [1] "Heart" "No"    "0.7"  
## [1] 1
## [1] 2
## [1] 3
## [1] 4
## [1] 5
## [1] 6
## [1] 7
## [1] 8
## [1] 9
## [1] 10
## [1] 11
## [1] 12
## [1] 13
## [1] 14
## [1] 15
## [1] 16
## [1] 17
## [1] 18
## [1] 19
## [1] 20
## [1] 21
## [1] 22
## [1] 23
## [1] 24
## [1] 25
## [1] 26
## [1] 27
## [1] 28
## [1] 29
## [1] 30
## [1] "Heart" "No"    "0.9"  
## [1] 1
## [1] 2
## [1] 3
## [1] 4
## [1] 5
## [1] 6
## [1] 7
## [1] 8
## [1] 9
## [1] 10
## [1] 11
## [1] 12
## [1] 13
## [1] 14
## [1] 15
## [1] 16
## [1] 17
## [1] 18
## [1] 19
## [1] 20
## [1] 21
## [1] 22
## [1] 23
## [1] 24
## [1] 25
## [1] 26
## [1] 27
## [1] 28
## [1] 29
## [1] 30
## [1] "diabetes-vid" "Yes"          "0.1"         
## [1] 1
## [1] 2
## [1] 3
## [1] 4
## [1] 5
## [1] 6
## [1] 7
## [1] 8
## [1] 9
## [1] 10
## [1] 11
## [1] 12
## [1] 13
## [1] 14
## [1] 15
## [1] 16
## [1] 17
## [1] 18
## [1] 19
## [1] 20
## [1] 21
## [1] 22
## [1] 23
## [1] 24
## [1] 25
## [1] 26
## [1] 27
## [1] 28
## [1] 29
## [1] 30
## [1] "diabetes-vid" "Yes"          "0.3"         
## [1] 1
## [1] 2
## [1] 3
## [1] 4
## [1] 5
## [1] 6
## [1] 7
## [1] 8
## [1] 9
## [1] 10
## [1] 11
## [1] 12
## [1] 13
## [1] 14
## [1] 15
## [1] 16
## [1] 17
## [1] 18
## [1] 19
## [1] 20
## [1] 21
## [1] 22
## [1] 23
## [1] 24
## [1] 25
## [1] 26
## [1] 27
## [1] 28
## [1] 29
## [1] 30
## [1] "diabetes-vid" "Yes"          "0.5"         
## [1] 1
## [1] 2
## [1] 3
## [1] 4
## [1] 5
## [1] 6
## [1] 7
## [1] 8
## [1] 9
## [1] 10
## [1] 11
## [1] 12
## [1] 13
## [1] 14
## [1] 15
## [1] 16
## [1] 17
## [1] 18
## [1] 19
## [1] 20
## [1] 21
## [1] 22
## [1] 23
## [1] 24
## [1] 25
## [1] 26
## [1] 27
## [1] 28
## [1] 29
## [1] 30
## [1] "diabetes-vid" "Yes"          "0.7"         
## [1] 1
## [1] 2
## [1] 3
## [1] 4
## [1] 5
## [1] 6
## [1] 7
## [1] 8
## [1] 9
## [1] 10
## [1] 11
## [1] 12
## [1] 13
## [1] 14
## [1] 15
## [1] 16
## [1] 17
## [1] 18
## [1] 19
## [1] 20
## [1] 21
## [1] 22
## [1] 23
## [1] 24
## [1] 25
## [1] 26
## [1] 27
## [1] 28
## [1] 29
## [1] 30
## [1] "diabetes-vid" "Yes"          "0.9"         
## [1] 1
## [1] 2
## [1] 3
## [1] 4
## [1] 5
## [1] 6
## [1] 7
## [1] 8
## [1] 9
## [1] 10
## [1] 11
## [1] 12
## [1] 13
## [1] 14
## [1] 15
## [1] 16
## [1] 17
## [1] 18
## [1] 19
## [1] 20
## [1] 21
## [1] 22
## [1] 23
## [1] 24
## [1] 25
## [1] 26
## [1] 27
## [1] 28
## [1] 29
## [1] 30
## [1] "diabetes-vid" "No"           "0.1"         
## [1] 1
## [1] 2
## [1] 3
## [1] 4
## [1] 5
## [1] 6
## [1] 7
## [1] 8
## [1] 9
## [1] 10
## [1] 11
## [1] 12
## [1] 13
## [1] 14
## [1] 15
## [1] 16
## [1] 17
## [1] 18
## [1] 19
## [1] 20
## [1] 21
## [1] 22
## [1] 23
## [1] 24
## [1] 25
## [1] 26
## [1] 27
## [1] 28
## [1] 29
## [1] 30
## [1] "diabetes-vid" "No"           "0.3"         
## [1] 1
## [1] 2
## [1] 3
## [1] 4
## [1] 5
## [1] 6
## [1] 7
## [1] 8
## [1] 9
## [1] 10
## [1] 11
## [1] 12
## [1] 13
## [1] 14
## [1] 15
## [1] 16
## [1] 17
## [1] 18
## [1] 19
## [1] 20
## [1] 21
## [1] 22
## [1] 23
## [1] 24
## [1] 25
## [1] 26
## [1] 27
## [1] 28
## [1] 29
## [1] 30
## [1] "diabetes-vid" "No"           "0.5"         
## [1] 1
## [1] 2
## [1] 3
## [1] 4
## [1] 5
## [1] 6
## [1] 7
## [1] 8
## [1] 9
## [1] 10
## [1] 11
## [1] 12
## [1] 13
## [1] 14
## [1] 15
## [1] 16
## [1] 17
## [1] 18
## [1] 19
## [1] 20
## [1] 21
## [1] 22
## [1] 23
## [1] 24
## [1] 25
## [1] 26
## [1] 27
## [1] 28
## [1] 29
## [1] 30
## [1] "diabetes-vid" "No"           "0.7"         
## [1] 1
## [1] 2
## [1] 3
## [1] 4
## [1] 5
## [1] 6
## [1] 7
## [1] 8
## [1] 9
## [1] 10
## [1] 11
## [1] 12
## [1] 13
## [1] 14
## [1] 15
## [1] 16
## [1] 17
## [1] 18
## [1] 19
## [1] 20
## [1] 21
## [1] 22
## [1] 23
## [1] 24
## [1] 25
## [1] 26
## [1] 27
## [1] 28
## [1] 29
## [1] 30
## [1] "diabetes-vid" "No"           "0.9"         
## [1] 1
## [1] 2
## [1] 3
## [1] 4
## [1] 5
## [1] 6
## [1] 7
## [1] 8
## [1] 9
## [1] 10
## [1] 11
## [1] 12
## [1] 13
## [1] 14
## [1] 15
## [1] 16
## [1] 17
## [1] 18
## [1] 19
## [1] 20
## [1] 21
## [1] 22
## [1] 23
## [1] 24
## [1] 25
## [1] 26
## [1] 27
## [1] 28
## [1] 29
## [1] 30
\end{verbatim}

\hypertarget{experimentos-sugeridos-opciuxf3n-1}{%
\subsection{Experimentos Sugeridos (Opción
1)}\label{experimentos-sugeridos-opciuxf3n-1}}

\includegraphics{outputs/plots/plot_exp_1.jpg}

En primer lugar, se destaca una clara correlación entre la proporción de
valores faltantes (NANs) y el rendimiento del modelo. La evidencia
gráfica demuestra que a medida que esta proporción aumenta, el valor del
área bajo la curva (AUC) disminuye significativamente. Un ejemplo
ilustrativo es la comparación entre un 0,1 de NA's en los datos de
entrenamiento y una proporción de 0.9, donde el AUC es considerablemente
mayor en el primer caso.

En cuanto a la estrategia de manejo de valores faltantes, se observa que
el enfoque no imputado, permitiendo que el árbol de decisión gestione
los NANs, alcanza el máximo rendimiento AUC en los cuatro escenarios
analizados. Esto sugiere que, en estos conjuntos de datos específicos,
el manejo de valores faltantes por parte de los árboles de decisión
supera a la imputación con medidas como el promedio o la moda.

Otra observación pertinente recae en la profundidad máxima del árbol. En
los primeros cuatro escenarios, se alcanza el mejor AUC con una
profundidad máxima de alrededor de entre 5 y 10 niveles. Sin embargo, a
medida que esta profundidad aumenta, el rendimiento AUC tiende a
disminuir, lo cual podría atribuirse a problemas de sobreajuste
(overfiting), donde el árbol se adapta demasiado a los datos de
entrenamiento y, por ende, presenta un rendimiento deficiente en la
predicción.

Una excepción se presenta en los escenarios con una proporción de NA's
de 0.9, donde se alcanza el máximo rendimiento AUC en un nivel de
profundidad más elevado. Este fenómeno plantea interrogantes sobre la
causa subyacente y su relación con la alta proporción de valores
faltantes en este caso en particular.

En conclusión, en los tres distintos conjuntos de datos, emerge un
patrón claro que muestra una relación evidente:

A medida que la proporción de valores faltantes aumenta, el rendimiento
del modelo disminuye, lo cual se refleja en la reducción del valor del
área bajo la curva (AUC).

Este fenómeno podría atribuirse a la dificultad que enfrenta el modelo
al aprender de manera efectiva cuando se encuentran ausentes una
cantidad significativa de valores.

Una hipótesis intuitiva para explicar esta relación es que a medida que
la proporción de valores faltantes aumenta, el proceso de aprendizaje
del modelo se ve deteriorado.

Por ejemplo, consideremos un escenario en el que en los datos de
entrenamiento, se registra que un individuo murió por diabetes, pero los
valores de presión arterial y niveles de insulina presentan valores
faltantes. En esta situación, el modelo no puede aprovechar esa
observación completa para comprender los patrones asociados a las
personas muertas por causa de diabetes. Conforme la proporción de
valores faltantes aumenta, el modelo se enfrenta a una carencia crítica
de información para capturar y relacionar las características
característicamente asociadas con la condición en cuestión. En esencia,
a medida que la información disponible se vuelve más escasa debido a los
valores faltantes, la capacidad del modelo para generalizar y realizar
predicciones precisas se ve afectada negativamente.

El modelo lucha por identificar patrones y relaciones en los datos, lo
que lleva a un descenso en su rendimiento, como se evidencia en la
disminución del valor de AUC.

Por lo tanto, la presencia de una alta proporción de valores faltantes
puede limitar la habilidad del modelo para capturar de manera efectiva
las relaciones subyacentes y, en última instancia, conducir a un
deterioro en su capacidad de predicción.

\hypertarget{section}{%
\subsection{}\label{section}}

\end{document}
